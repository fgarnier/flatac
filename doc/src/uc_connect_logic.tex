\subsection{\tsl{How to :} Plug your own logic and semantic to \flatac ?}
The architecture of \flatac was made to be as generic as possible. One of our goals was that, whatever the complexity of your \tsl{\gls{ls}}, plugging it into \flatac has to be easy and efficient.

\newpar To comply with these aims, we designed a fairly simple architecture :\\
%\image{img/FrontEnd}{1}

\begin{center}% Graphic for TeX using PGF
% Title: /Users/maximegaudin/Documents/Professionel/Stages/Verimag/Developpement/flatac/doc/src/img/FrontEnd.dia
% Creator: Dia v0.97.1
% CreationDate: Wed May 18 14:12:05 2011
% For: maximegaudin
% \usepackage{tikz}
% The following commands are not supported in PSTricks at present
% We define them conditionally, so when they are implemented,
% this pgf file will use them.
\ifx\du\undefined
  \newlength{\du}
\fi
\setlength{\du}{15\unitlength}
{\scalefont{0.5}
\begin{tikzpicture}[scale=0.7]
\pgftransformxscale{1.000000}
\pgftransformyscale{-1.000000}
\definecolor{dialinecolor}{rgb}{0.000000, 0.000000, 0.000000}
\pgfsetstrokecolor{dialinecolor}
\definecolor{dialinecolor}{rgb}{1.000000, 1.000000, 1.000000}
\pgfsetfillcolor{dialinecolor}
\pgfsetlinewidth{0.100000\du}
\pgfsetdash{}{0pt}
\definecolor{dialinecolor}{rgb}{1.000000, 1.000000, 1.000000}
\pgfsetfillcolor{dialinecolor}
\fill (14.850000\du,5.150000\du)--(14.850000\du,6.550000\du)--(35.370000\du,6.550000\du)--(35.370000\du,5.150000\du)--cycle;
\definecolor{dialinecolor}{rgb}{0.000000, 0.000000, 0.000000}
\pgfsetstrokecolor{dialinecolor}
\draw (14.850000\du,5.150000\du)--(14.850000\du,6.550000\du)--(35.370000\du,6.550000\du)--(35.370000\du,5.150000\du)--cycle;
% setfont left to latex
\definecolor{dialinecolor}{rgb}{0.000000, 0.000000, 0.000000}
\pgfsetstrokecolor{dialinecolor}
\node at (25.110000\du,6.100000\du){semAndLogicFrontEnd};
\definecolor{dialinecolor}{rgb}{1.000000, 1.000000, 1.000000}
\pgfsetfillcolor{dialinecolor}
\fill (14.850000\du,6.550000\du)--(14.850000\du,6.950000\du)--(35.370000\du,6.950000\du)--(35.370000\du,6.550000\du)--cycle;
\definecolor{dialinecolor}{rgb}{0.000000, 0.000000, 0.000000}
\pgfsetstrokecolor{dialinecolor}
\draw (14.850000\du,6.550000\du)--(14.850000\du,6.950000\du)--(35.370000\du,6.950000\du)--(35.370000\du,6.550000\du)--cycle;
\definecolor{dialinecolor}{rgb}{1.000000, 1.000000, 1.000000}
\pgfsetfillcolor{dialinecolor}
\fill (14.850000\du,6.950000\du)--(14.850000\du,10.350000\du)--(35.370000\du,10.350000\du)--(35.370000\du,6.950000\du)--cycle;
\definecolor{dialinecolor}{rgb}{0.000000, 0.000000, 0.000000}
\pgfsetstrokecolor{dialinecolor}
\draw (14.850000\du,6.950000\du)--(14.850000\du,10.350000\du)--(35.370000\du,10.350000\du)--(35.370000\du,6.950000\du)--cycle;
% setfont left to latex
\definecolor{dialinecolor}{rgb}{0.000000, 0.000000, 0.000000}
\pgfsetstrokecolor{dialinecolor}
\node[anchor=west] at (15.000000\du,7.650000\du){+<<virtual>> getEntryPointAbstraction(): 'a};
% setfont left to latex
\definecolor{dialinecolor}{rgb}{0.000000, 0.000000, 0.000000}
\pgfsetstrokecolor{dialinecolor}
\node[anchor=west] at (15.000000\du,8.450000\du){+<<virtual>> getEntryPointPrecondition(): 'a};
% setfont left to latex
\definecolor{dialinecolor}{rgb}{0.000000, 0.000000, 0.000000}
\pgfsetstrokecolor{dialinecolor}
\node[anchor=west] at (15.000000\du,9.250000\du){+<<virtual>> next('a,string,stmtkind): ('a * string)};
% setfont left to latex
\definecolor{dialinecolor}{rgb}{0.000000, 0.000000, 0.000000}
\pgfsetstrokecolor{dialinecolor}
\node[anchor=west] at (15.000000\du,10.050000\du){+<<virtual>> pretty('a): string};
\definecolor{dialinecolor}{rgb}{1.000000, 1.000000, 1.000000}
\pgfsetfillcolor{dialinecolor}
\fill (33.070000\du,4.450000\du)--(33.070000\du,5.450000\du)--(35.770000\du,5.450000\du)--(35.770000\du,4.450000\du)--cycle;
\pgfsetdash{{1.000000\du}{1.000000\du}}{0\du}
\pgfsetdash{{0.300000\du}{0.300000\du}}{0\du}
\definecolor{dialinecolor}{rgb}{0.000000, 0.000000, 0.000000}
\pgfsetstrokecolor{dialinecolor}
\draw (33.070000\du,4.450000\du)--(33.070000\du,5.450000\du)--(35.770000\du,5.450000\du)--(35.770000\du,4.450000\du)--cycle;
% setfont left to latex
\definecolor{dialinecolor}{rgb}{0.000000, 0.000000, 0.000000}
\pgfsetstrokecolor{dialinecolor}
\node[anchor=west] at (33.370000\du,5.150000\du){'a:};
\pgfsetlinewidth{0.100000\du}
\pgfsetdash{}{0pt}
\definecolor{dialinecolor}{rgb}{1.000000, 1.000000, 1.000000}
\pgfsetfillcolor{dialinecolor}
\fill (42.850000\du,18.200000\du)--(42.850000\du,19.600000\du)--(51.535000\du,19.600000\du)--(51.535000\du,18.200000\du)--cycle;
\definecolor{dialinecolor}{rgb}{0.000000, 0.000000, 0.000000}
\pgfsetstrokecolor{dialinecolor}
\draw (42.850000\du,18.200000\du)--(42.850000\du,19.600000\du)--(51.535000\du,19.600000\du)--(51.535000\du,18.200000\du)--cycle;
\definecolor{dialinecolor}{rgb}{1.000000, 1.000000, 1.000000}
\pgfsetfillcolor{dialinecolor}
\fill (42.850000\du,17.300000\du)--(42.850000\du,18.200000\du)--(44.350000\du,18.200000\du)--(44.350000\du,17.300000\du)--cycle;
\definecolor{dialinecolor}{rgb}{0.000000, 0.000000, 0.000000}
\pgfsetstrokecolor{dialinecolor}
\draw (42.850000\du,17.300000\du)--(42.850000\du,18.200000\du)--(44.350000\du,18.200000\du)--(44.350000\du,17.300000\du)--cycle;
% setfont left to latex
\definecolor{dialinecolor}{rgb}{0.000000, 0.000000, 0.000000}
\pgfsetstrokecolor{dialinecolor}
\node[anchor=west] at (43.150000\du,19.095000\du){YOUR Logic \& Semantic};
\pgfsetlinewidth{0.100000\du}
\pgfsetdash{}{0pt}
\definecolor{dialinecolor}{rgb}{1.000000, 1.000000, 1.000000}
\pgfsetfillcolor{dialinecolor}
\fill (10.995000\du,15.830000\du)--(10.995000\du,17.230000\du)--(39.215000\du,17.230000\du)--(39.215000\du,15.830000\du)--cycle;
\definecolor{dialinecolor}{rgb}{0.000000, 0.000000, 0.000000}
\pgfsetstrokecolor{dialinecolor}
\draw (10.995000\du,15.830000\du)--(10.995000\du,17.230000\du)--(39.215000\du,17.230000\du)--(39.215000\du,15.830000\du)--cycle;
% setfont left to latex
\definecolor{dialinecolor}{rgb}{0.000000, 0.000000, 0.000000}
\pgfsetstrokecolor{dialinecolor}
\node at (25.105000\du,16.780000\du){fooAbstraction};
\definecolor{dialinecolor}{rgb}{1.000000, 1.000000, 1.000000}
\pgfsetfillcolor{dialinecolor}
\fill (10.995000\du,17.230000\du)--(10.995000\du,17.630000\du)--(39.215000\du,17.630000\du)--(39.215000\du,17.230000\du)--cycle;
\definecolor{dialinecolor}{rgb}{0.000000, 0.000000, 0.000000}
\pgfsetstrokecolor{dialinecolor}
\draw (10.995000\du,17.230000\du)--(10.995000\du,17.630000\du)--(39.215000\du,17.630000\du)--(39.215000\du,17.230000\du)--cycle;
\definecolor{dialinecolor}{rgb}{1.000000, 1.000000, 1.000000}
\pgfsetfillcolor{dialinecolor}
\fill (10.995000\du,17.630000\du)--(10.995000\du,21.030000\du)--(39.215000\du,21.030000\du)--(39.215000\du,17.630000\du)--cycle;
\definecolor{dialinecolor}{rgb}{0.000000, 0.000000, 0.000000}
\pgfsetstrokecolor{dialinecolor}
\draw (10.995000\du,17.630000\du)--(10.995000\du,21.030000\du)--(39.215000\du,21.030000\du)--(39.215000\du,17.630000\du)--cycle;
% setfont left to latex
\definecolor{dialinecolor}{rgb}{0.000000, 0.000000, 0.000000}
\pgfsetstrokecolor{dialinecolor}
\node[anchor=west] at (11.145000\du,18.330000\du){+getEntryPointAbstraction(): fooAbstractionType};
% setfont left to latex
\definecolor{dialinecolor}{rgb}{0.000000, 0.000000, 0.000000}
\pgfsetstrokecolor{dialinecolor}
\node[anchor=west] at (11.145000\du,19.130000\du){+getEntryPointPrecondition(): fooAbstractionType};
% setfont left to latex
\definecolor{dialinecolor}{rgb}{0.000000, 0.000000, 0.000000}
\pgfsetstrokecolor{dialinecolor}
\node[anchor=west] at (11.145000\du,19.930000\du){+next(fooAbstractionType,string,stmtkind): (fooAbstractionType * string)};
% setfont left to latex
\definecolor{dialinecolor}{rgb}{0.000000, 0.000000, 0.000000}
\pgfsetstrokecolor{dialinecolor}
\node[anchor=west] at (11.145000\du,20.730000\du){+pretty(fooAbstractionType): string};
\pgfsetlinewidth{0.100000\du}
\pgfsetdash{}{0pt}
\pgfsetmiterjoin
\pgfsetbuttcap
{
\definecolor{dialinecolor}{rgb}{0.000000, 0.000000, 0.000000}
\pgfsetfillcolor{dialinecolor}
% was here!!!
\definecolor{dialinecolor}{rgb}{0.000000, 0.000000, 0.000000}
\pgfsetstrokecolor{dialinecolor}
\draw (25.110000\du,10.400323\du)--(25.110000\du,13.115162\du)--(25.105000\du,13.115162\du)--(25.105000\du,15.830000\du);
}
\definecolor{dialinecolor}{rgb}{0.000000, 0.000000, 0.000000}
\pgfsetstrokecolor{dialinecolor}
\draw (25.110000\du,11.312127\du)--(25.110000\du,13.115162\du)--(25.105000\du,13.115162\du)--(25.105000\du,15.830000\du);
\pgfsetmiterjoin
\definecolor{dialinecolor}{rgb}{1.000000, 1.000000, 1.000000}
\pgfsetfillcolor{dialinecolor}
\fill (25.510000\du,11.312127\du)--(25.110000\du,10.512127\du)--(24.710000\du,11.312127\du)--cycle;
\pgfsetlinewidth{0.100000\du}
\pgfsetdash{}{0pt}
\pgfsetmiterjoin
\definecolor{dialinecolor}{rgb}{0.000000, 0.000000, 0.000000}
\pgfsetstrokecolor{dialinecolor}
\draw (25.510000\du,11.312127\du)--(25.110000\du,10.512127\du)--(24.710000\du,11.312127\du)--cycle;
% setfont left to latex
\pgfsetlinewidth{0.100000\du}
\pgfsetdash{}{0pt}
\pgfsetdash{}{0pt}
\pgfsetbuttcap
{
\definecolor{dialinecolor}{rgb}{0.000000, 0.000000, 0.000000}
\pgfsetfillcolor{dialinecolor}
% was here!!!
\pgfsetarrowsend{stealth}
\definecolor{dialinecolor}{rgb}{0.000000, 0.000000, 0.000000}
\pgfsetstrokecolor{dialinecolor}
\draw (39.215000\du,18.930000\du)--(42.850000\du,18.900000\du);
}
\end{tikzpicture}
}
\end{center}

The \tsl{front-end} acts as an interface between the \gls{eCFG} algorithms and data structures, this way, even if the methods of the \tsl{front-end} are constrained, you can choose the way to implement your own \tsl{\gls{ls}}.

\subsubsection{Front-end}
Let's study in detail all the methods of the \tsl{front-end}. Let \ttt{'a} be the parameter of the \tsl{front-end} class :

\code[Caml]{getEntryPointAbstraction}{getEntryPointAbstraction_sig.ml}
This method provides the abstract interpretation of the entry point, which is the thus, the state of the \tsl{eCFG}'s root.

\code[Caml]{getEntryPointPrecondition}{getEntryPointPrecondition_sig.ml}
This method provides the precondition of the entry point, which is the thus, the precondition of the \tsl{eCFG}'s root.

\code[Caml]{next}{next_sig.ml}
Return the couple of (Abstraction, CounterAutomataExpression) of the next state.

\code[Caml]{isErrorState}{isErrorState_sig.ml}
This method returns true if the given abstraction is an error state. It returns false otherwise.

\code[Caml]{Pretty}{pretty_sig.ml}
This methods convert the given abstract interpretation to displayable string. It's mainly used for debugging.


