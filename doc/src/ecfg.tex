The \gls{eCFG} is the internal way for \flatac to represent the \tsl{Frama-C} 
\gls{CFG} plus our own datas such as :
\begin{itemize}
  \item Statements and their abstraction,
  \item each transition label,
  \item value analysis \footnote{Not implemented yet.}.
\end{itemize}

To be generic, the module is parametrized by :
\begin{enumerate}
  \item The abstraction type
  \item The transition label type
\end{enumerate}

\image{img/ecfg}{1}

Because it encapsulates all the complexity, the only way to access to 
stored data is to use the visitor.

\subsection{eCFG's visitor}
The visitor return each node of the given eCFGs, one after another. Please do 
not assume any order.

The visitor method take three callbacks in parameter :
\begin{enumerate}
  \item Precallback : The callback function called before each \gls{eCFG} is 
    browsed.
  \item Callback : The callback function called at each node.
  \item Postcallback : The callback function called after each \gls{eCFG} is 
    browsed.
\end{enumerate}
