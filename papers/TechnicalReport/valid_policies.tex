\section{Validity policies}

In a previous section of this chapter, we mentionned that $\isvalid{\phi}{.}$ 
is a function if for every integer variable $i$, resp. pointer variable
$x$, and at each abstract state of the program there exists one and only one evaluation for $\isvalid{\phi}{i}$, resp. $\isvalid{\phi}{x}$.

Since the evaluation of the validity of each variable $i$, $\isvalid{\phi}{i}$, 
on any next state is deduced from the current state, the evaluation of
the validity of a variable only depends of the initial value of $\isvalid{\phi}{i}$
at the entry point of the considered function.

We introduce a policy for initializing the validity of each integer and each
pointer variable, and we defend why we fix such values.
We consider tree subset of variables :

\begin{itemize}
\item The set of global variables,
\item the set of parameters, also called "formals" in Cil,
\item the set of local variabes, stored on the stack.
\end{itemize}

Here is the initial value assigned to, w.l.o.g. $\isvalid{\Formula{\true}{\Emp}}{i}$ at the entry point of the function (TODO : Explique plus clairement ce
aue l'on entend par "point d'entree de la fonction etudiee"):

$$
\begin{array}{|l|l|}
\hline
\mbox{Location} & \mbox{Initial evaluation} \\
\hline
\mbox{Global}& \dk \\
\mbox{Parameters} & \dk \\
\mbox{Locals} & \false \\
\hline
\end{array}
$$
