%% This file contains some results related to properties satisfied between formulae that satisfy the entailement relation.

\begin{definition}[Size of a formula]
The size of a formula is defined as :


$$
\begin{array}{|cc|}
\hline
&\\
\sizef{\Formula{\true}{\Emp}}=1 & \Andpure{\Pointsto{x}{l}}{\phi}=1+\sizef{\phi}\\
\sizef{\exists l. \Andpure{\Pointsto{x}{l}}{\phi}} = 1+\sizef{\phi}& \sizef{\Sep{\phi}{\Alloc{l}} } = 1+\sizef{\phi} \\
\sizef{\Sep{\phi}{ \exists l. \Alloc{l}} } = 1+\sizef{\phi} & \\
&\\
\hline
\end{array}
$$
\end{definition}

\begin{lemma}
\label{lemma:entailsize}
Consider two SSL formulae in normal form, $\phi$ and $\psi$. If $\phi \vdash \psi$ then
$\PVar(\phi)=\PVar(\psi)$ and $\sizef{\phi}=\sizef{\psi}$.
\end{lemma}

\begin{proof}

We first start to prove that $ \PVar(\phi) \subseteq \PVar(\psi)$:

Assume that $\phi\vdash \psi$ and let us take $x\in \PVar(\phi)$. 
We have that $\Pointsto{x}{l}\in \Pp{\phi}$, with $l\in \LVar \cup \lbrace \nil \rbrace $.
$$
\forall i, \forall s , \forall h, s.t. <i,s,h>\models \phi \Rightarrow <i,s,h>\models \psi,
$$

therefore, following the semantic, we have that :

$$\psi=\Pointsto{x}{l}\wedge \psi^{\prime},$$ which implies that $x \in \PVar(\psi)$.

Let us prove the other inclusion, i.e. $ \PVar(\phi)\supseteq \PVar(\psi)$ :

Let be $\phi = \Formula{\bigwedge_{i=1}^n \Pointsto{x_i}{l_i}}{\sigma_{\phi}}$ and $\psi$, with $\phi \vdash \psi$. 
Let us suppose that there exists some $y\in \PVar(\psi)$ but $y\not\in \PVar(\phi)$. We have that the litteral $\Pointsto{y}{l}$ is in $\Pp{\psi}$, where $l\in \LVar \cup \lbrace  \nil \rbrace.$

Consider that the model $<i,s,h>$ is such that
$<i,s,h>\models \phi$. The relation $<i,s,h>\models \phi$ implies that $\bigwedge_{k=1}^n s(x_k) = i(l_k)$. 
Now consider the two functions $i,s$ such that for any integer $k$ in $\lbrace1,\ldots,n \rbrace$ we have $s(x_k)=i$, and that $s(y)=l^{\prime}$ where $l \neq l^{\prime}$.
In this case, we have that $<i,s,h>\models \phi$, however $<i,s,h> \not\models \psi$, which is in contradiction with the hypothesis $\phi \vdash  \psi$, hence
the other inclusion.


Both formulae defines the same heap, therefore, $\sizef{\Sp{\phi}}=\sizef{\Sp{\psi}}$, and we deduce from this observation and the previous result that :

$$\phi \vdash \psi \Rightarrow \sizef{\phi}=\sizef{\psi}.$$

\end{proof}


\begin{lemma}
If two SSL formulae $\phi$ and $\psi$ satisfy the relation $\phi \models \psi$ 
then $\PVar(\phi)=\PVar(\psi)$ and $\sizef{\phi}=\sizef{\psi}$.
\end{lemma}

\begin{proof}

Let us show that $\PVar(\phi)=\PVar(\psi)$.
We use the same arguments as in the proof of lemma \ref{lemma:entailsize}
to show the inclusion $\subseteq$.

Let's now show that $\PVar(\phi) \supseteq \PVar(\psi)$.
To to so, we suppose that there exists some variable $y \in \PVar(\psi)$, and
that the hypothesis :
\begin{enumerate}
\item $\phi \models \psi$,
\item $y\not\in \PVar(\phi)$,
\end{enumerate}
lead to a contradiction.
%Let be $\phi = \Formula{\bigwedge_{i=1}^n \Pointsto{x_i}{l_i}}{\sigma_{\phi}}$ and $\psi$, with $\phi \vdash \psi$. 
The fact that $y\not\in\PVar(\phi)$ means that $\Pointsto{y}{l}\not\in \Pp{\phi}$ $\forall l\in\LVar$, whereas there exists some $l\in \LVar \cup \lbrace \nil \rbrace$ such that $\Pointsto{y}{l}\in \Pp{\psi}$.
Consider a model $<i,s,h>$ such that $<i,s,h>\models \phi$. 
The interpretations $i$ and $s$ satisfy that for all $l\in \LVar$, $s(y)\neq i(l)$,
therefore $<i,s,h>\not\models \psi$.

At this stage, we proved that $\PVar(\phi)=\PVar(\psi)$, therefore $\sizef{\psi}=\sizef{\phi}$.
\end{proof}


\begin{definition}[Subformula]
A formula $\phi^{\prime}=Q^{\prime} \Formula{\pi{\prime}}{\sigma^{\prime}}$ is a subformula of $\phi=Q \Formula{\pi}{\sigma}$, noted $\phi \subf \phi^{\prime}$, if for all $l\in Q^{\prime}$, $l\in Q$ and for all $\Pointsto{x}{l}\in \pi^{\prime}, \Pointsto{x}{l}\in \pi$ and
if $\Alloc{l}\in \sigma^{\prime}$ then $\Alloc{l} \in \sigma$.
\end{definition}


\begin{lemma}
\label{lemma:modelsubfcns}
Let $\phi$ and $\psi$ two SSL formulae such that $\phi \models \psi$. Consider $\phi^{\prime}$ and $\psi^{\prime}$ such that $\phi \subf \phi^{\prime}$ and $\psi \subf \psi^{\prime}$.
We have the following equivalence :

$$\psi^{\prime} \models \phi^{\prime} \Leftrightarrow \left \lbrace \begin{array}{l} \PVar(\phi^{\prime}) = \PVar(\psi^{\prime}) \\ 
\wedge \sizef{\psi^{\prime}}=\sizef{\phi^{\prime}}. 
\end{array} \right .
$$

\end{lemma}

\begin{lemma}
\label{lemma:completeness}
For every tuple of formula in normal form, $\phi$ and $\psi$, such that $\phi \models \psi$, there exists
$\phi^{\prime}$, resp. $\psi^{\prime}$, such that $\phi \subf \phi^{\prime}$, resp. $\psi \subf \psi^{\prime}$, that satisfy the conditions below :
\begin{itemize}
\item $\phi^{\prime} \models \psi^{\prime}$,
\item $\phi \vdash \psi \leadsto \phi^{\prime} \vdash \psi^{\prime}$,
\item $\sizef{\phi}=\sizef{\psi}>\sizef{\phi^{\prime}}=\sizef{\psi^{\prime}}.$
\end{itemize}
\end{lemma}

\begin{proof}

We prove lemma \ref{lemma:completeness}~ by induction on the size of the formulae.
Let us set that the following $(\Formula{\true}{\Emp}) \models (\Formula{\true}{\Emp})$ is
true.
The only formula which size is one is indeed $(\Formula{\true}{\Emp})$.
We need to show that the tree conditions of lemma \ref{lemma:modelsubfcns}~ are 
always statisfied when $\phi$ and $\psi$ have a size which is equal to $2$.

\textbf{Base case :}
The formulae which size equals $2$ are the formula that either have on litteral
in the pure part or have one allocated cell declarated within their spacial part.

The considerated formulae, without existential quantificator are :
\begin{itemize}
\item $\Formula{\Pointsto{x}{l}}{\Emp}$,
\item $\Formula{\true}{\Alloc{l}}$ which is the smallest garbage formula,
\end{itemize}
where $x$ is a free pointer variable and $l$ is a location variable, possibly bounded.

Consider the first case. Let's suppose that $\Formula{\Pointsto{x}{l}}{\Emp}\models \psi$. Using lemma \ref{lemma:modelsubfcns}, we deduce that $\PVar(\psi)=\lbrace x\rbrace$, and that $\psi=\Formula{\Pointsto{x}{l}}{\Emp}$ if $l\in \FVars{\phi}$, or
$\psi=\Formula{\exists l^{\prime}.\Pointsto{x}{l^{\prime}}}{\Emp}$.
In both cases, the term $\phi \vdash \psi$ is reduced to $\Formula{\true}{\Emp}$, using the rule $r_1$ in the first case and the rule $r_4$ in the second case.

Let's now consider the secon case, that is to say the set of formulae $\psi$ such that $\phi=\Formula{\true}{\Alloc{l}}$. Let be $\psi$ such that $ \phi \models \psi$. Using the lemma  \ref{lemma:modelsubfcns}, we know that $\psi=\Formula{\true}{\Alloc{l^{\prime}}}$, for some $l^{\prime} \in \LVar$.
Here both $l$ and $l^{\prime}$  can be free or bounded. In every case, the formula
$\Formula{\true}{\Alloc{l}} \vdash \Formula{\true}{\Alloc{l^{\prime}}}$ 
can be reduced to $\Formula{\true}{\Emp} \vdash \Formula{\true}{\Emp}$, using 
the rules $r_2,r_4,r_4,r_6,$ depending on the cases.  


\textbf{Induction :}

We suppose that the three properties are satified for all tuuple of formula $\phi, \psi$ such that $\phi \models \psi$ and such that $\sizef{\phi}\leq n$.

Let  two formulae $\phi$ and $\psi$ be such that $\phi \models \psi$ and
 $\sizef{\phi}=n+1$.

\textbf{Elimination of $\Pointsto{x}{l}$ litterals :}

Let's suppose that $\phi \models \psi$ and that : $\Pointsto{x}{l}\in \Pp{\phi}$.
Lemma \ref{lemma:modelsubfcns} entails that $x\in \PVar(\psi)$. 
%For any model $<i,s,h>\models \phi_1$, we have, by definition, that $<i,s,h>\models \phi_2$.  
Therefore, in $\psi$ there is a location variable $l^{\prime}$ such that $s(x)=i(l^{\prime})$. 
Let us first consider the case where $l$ is a free variable of $\phi$ :

\begin{itemize}
\item If $l^{\prime}$ is free then we have that $l^{\prime}\seq l$, then in this case $\Pointsto{x}{l}$ appears in both side of the entailement symbol. The rule $r_1$ can be use to eliminate them.
\item If $l^{\prime}$ is existentially quantified (in $\psi$), then we have that : $\exists l^{\prime}. \Pointsto{x}{l^{\prime}\in\Pp{\psi}}$. By applying the rule $r_3$ then the rule $r_1$, we eliminate the litteral $\Pointsto{x}{l}$ from
the lhs and the $\exists l^{\prime} \Pointsto{x}{l^{\prime}}$ from the right hand side.
\end{itemize}

We now deal with the case where the variable $l$ of $\phi$ is existentially quantified. Using the semantic of the $\models$ relation, we get that $s(x)=\Substin{i}{l}{\lambda}$ for some $\lambda \in \Addr$. In the same vein, as $<i,s,h> \models \psi$, we get that $\exists l^{\prime}\in \Pp{\psi}$ with $s(x)=\Substin{i}{l^{\prime}}{\lambda}$.
 The rule $r_4$ can eliminate the litterals in both part of the entailement symbol.

\textbf{Elimination of $\Pointstonil{x}$ litterals :}
If the litteral
$\Pointstonil{x} \in \Pp{\phi}$ then all model $<i,s,h>$ of $\phi$ satisfy $s(x)=\nil$. Therefore, for all $\phi \vdash \psi$, $s(x)=\nil$ must be expressed within the formula $\psi$, therefore $\Pointstonil{x}\in\Pp{\psi}$. By applying the rule $r_1$ ($\nil$ is a pointer variable), one eliminates both litteral form each side of the entailement symbol.

At this stage, we reduced the expression $\phi \vdash \psi$ to $\phi^{\prime} \vdash \psi^{\prime}$, where $\sizef{\phi^{\prime}}=\sizef{\psi^{\prime}}=n$, and where
$\PVar(\phi^{\prime})=\PVar(\psi^{\prime})=\PVar(\phi)/\lbrace x \rbrace$, hence
by lemma \ref{lemma:modelsubfcns}, we get that $\phi^{\prime}\models\psi^{\prime}$, therefore the induction hypothesis is met for $\phi^{\prime}\vdash \psi^{\prime}$.

 
\textbf{Elimination of the $\Alloc{l}$ litterals :}  

Let us consider all the different cases possible, liste bellow :

\begin{itemize}
\item $\exists l.\Alloc{l} \in \phi$, and $\Pointsto{x}{l}\not\in \Pp{\phi}, \forall x\in \PVar$. This formula contains some garbage.

\item $\Pointsto{x}{l}\in \phi$ and $\Alloc{l}\in \phi$. This implies
that there exists some $l^{\prime}\in LVar(\psi)$ s.t. $\Pointsto{x}{l^{\prime}}$.
\end{itemize}

Let's deal with the first case :

Consider that $ \exists l. \Alloc{l} $ and $\Pointsto{x}{l}\not\in \Pp{\phi}, \forall x\in\PVar$ implies that there
exists $l^{\prime} \in Vars(\psi)$ s.t. $i(l^{\prime})=i(l)$ and for all $x\in \PVar (\psi)$ $s(x)\neq i(l^{\prime})$, therefore $\psi$ contains no litterals $\Pointsto{x}{l^{\prime}}$, however it contains a litteral $\exists l^{\prime} \Alloc{l^{\prime}}$. Both of the $\Alloc{.}$ litterals can be eliminated using 
the rule $r_5$ if $l\in \FVars{\phi_1}$ or using $r_6$ is $l\not\in\FVars{\phi_1}$.  

And now, consider the second case :

\begin{itemize}
\item If $l^{\prime}\in \FVars{\psi}$, then $l^{\prime} \seq l$. In this case the
rule $r_2$ can eliminate both litteral $\Alloc{l}$.
\item If $l^{\prime}$ is existentially quantified in $\psi$, then apply rule $r_3$ or $r_4$ (depending on the fact that $l$ is a free variable of $\phi_1$, or not), then apply rule $r_2$, that eliminates both litterals $\Alloc{l}$.
\end{itemize}



At this stage, we reduced the expression $\phi \vdash \psi$ to $\phi^{\prime} \vdash \psi^{\prime}$, where $\sizef{\phi^{\prime}}=\sizef{\psi^{\prime}}=n$, and where
$\PVar(\phi^{\prime})=\PVar(\psi^{\prime})=\PVar(\phi)$, thus
by lemma \ref{lemma:modelsubfcns}, we get that $\phi^{\prime}\models\psi^{\prime}$, therefore the induction hypothesis is satified for $\phi^{\prime}\vdash \psi^{\prime}$.

This ends the proof of the completeness of the inference system. We showed, that
for any two formulae $\phi \models \psi$, there exists a reduction path that
leads to the term $\Formula{\true}{\Emp} \vdash \Formula{\true}{\Emp}$.
\end{proof}

The theorem \ref{th:completeness} is a direct consequence of lemma \ref{lemma:completeness}.
