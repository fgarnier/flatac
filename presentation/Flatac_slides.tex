%\newtheorem{definition}{Definition}
\newtheorem{lemma}{Lemma}
\newtheorem{theorem}{Theorem}
\newtheorem{proposition}{Proposition}
\newtheorem{corollary}{Corollary}
\newtheorem{conjecture}{Conjecture}
%\newtheorem{claim}{Claim}
%\newtheorem{property}{Property}
\newtheorem{example}{Example}

% Marius commands
\newcommand{\Z}{{\mathbb Z}}
\newcommand{\N}{{\mathbb N}}
\newcommand{\relmat}[1] { \{\!\!\{#1\}\!\!\}}
\newcommand{\matrel}[1] { [\!\![#1]\!\!]}
\newcommand{\key}[1]{\underline{#1}}

% semantic stuff
\newcommand{\bi}[1]{{}^\omega\!{#1}^\omega}
\newcommand{\rbr}{{\bf ]\!]}}
\newcommand{\lbr}{{\bf [\![}}
\newcommand{\sem}[1]{\lbr #1 \rbr}
\newcommand{\model}[2]{\sem{#1}_{#2}}
\newcommand{\lfp}[1]{\mbox{fix}\ {#1}}
\newcommand{\wpr}{\widetilde{pre}}
\newcommand{\cceil}[1]{\lfloor #1 \rfloor + 1}

\renewcommand{\vec}[1]{{\bf {#1}}}

% alias
\newcommand{\alias}{\Diamond}
\newcommand{\may}[2]{{\langle #1 \rangle}{#2}}
\newcommand{\must}[2]{{\lbrack #1 \rbrack}{#2}}
\newcommand{\defd}[1]{\delta({#1})}

% logic
\newcommand{\pal}{$pAL$}
\newcommand{\wal}{$wAL$}
\newcommand{\kal}{$\kappa AL$}

% cardinal, degree, ...
\newcommand{\sgn}{\mbox{sgn}}
%\newcommand{\gcd}{\mbox{gcd}}
\newcommand{\lcm}{\mbox{lcm}}
\newcommand{\len}[1]{{|{#1}|}}
\newcommand{\card}[1]{{|\!|{#1}|\!|}}
\newcommand{\minimum}[2]{\mbox{min}({#1},{#2})}
\newcommand{\maximum}[2]{\mbox{max}({#1},{#2})}
\newcommand{\minset}[2]{\mbox{min}\{{#1}\ |\ {#2}\}}
\newcommand{\maxset}[2]{\mbox{max}\{{#1}\ |\ {#2}\}}
\newcommand{\divs}[1]{\mbox{div}({#1})}

% automata and languages
\newcommand{\auth}[1]{{\mathcal A}_{#1}}
\newcommand{\lang}[1]{{\mathcal L}({#1})}
\newcommand{\runs}[1]{{\mathcal R}({#1})}
\newcommand{\val}[1]{{\mathcal V}({#1})}
\newcommand{\trace}[1]{{Tr}({#1})}
\newcommand{\state}[1]{{\Sigma}({#1})}
\newcommand{\trans}[1]{{\Theta}({#1})}
\newcommand{\access}[3]{{\mathcal L}_{{#1},{#2}}({#3})}
\newcommand{\eats}{\rightarrow}
\newcommand{\Eats}[1]{\stackrel{#1}{\longrightarrow}}
\newcommand{\arrow}[2]{\xrightarrow[{\scriptstyle #2}]{{\scriptstyle #1}}}
\newcommand{\larrow}[2]{\xleftarrow[{\scriptstyle #2}]{{\scriptstyle #1}}} 

% numbers
\newcommand{\nat}{{\bf \mathbb{N}}}
\newcommand{\zed}{{\bf \mathbb{Z}}}
\newcommand{\bool}{{\bf \mathbb{B}}}
\newcommand{\rat}{{\bf \mathbb{Q}}}
\newcommand{\real}{{\bf \mathbb{R}}}
\newcommand{\linzed}{{\bf \mathcal{L}\mathbb{Z}}}

% powerset
\newcommand{\pow}[1]{{\mathcal P}(#1)}
\newcommand{\pown}[2]{{\mathcal P}^{#1}(#2)}

% by definition
\newcommand{\bydef}{\stackrel{\Delta}{=}}

% projection, lifting
\newcommand{\proj}[2]{{#1}\!\!\downarrow_{{#2}}}
\newcommand{\lift}[2]{{#1}\!\!\uparrow_{{#2}}}

% forward, backward
\newcommand{\forw}[1]{\stackrel{\rightarrow}{#1}}
\newcommand{\back}[1]{\stackrel{\leftarrow}{#1}}

% equation figures
\newcommand{\eqnfigx}[1]{
\begin{minipage}{3in}
\begin{center}
\begin{eqnarray*}#1
\end{eqnarray*}
\end{center}
\end{minipage}}

\newcommand{\twoeqnfigx}[2]{
\eqnfigx{#1}
\eqnfigx{#2}
}

\newcommand{\threeeqnfigx}[3]{
\eqnfigx{#1}\\
\eqnfigx{#2}\eqnfigx{#3}
}

\newcommand{\eqnfig}[1]{
\begin{minipage}{3in}
\begin{center}
\begin{eqnarray}#1
\end{eqnarray}
\end{center}
\end{minipage}}

\newcommand{\twoeqnfig}[2]{
\eqnfig{#1}
\eqnfig{#2}
}

% transitions and SOS rules
\newcommand{\storetrans}[1]{\stackrel{#1}{\leadsto}}
\newcommand{\storetransx}[1]{\stackrel{#1}{\leadsto^*}}
\newcommand{\storetransprime}[3]{{#1} \storetrans{#2} {#3}}
\newcommand{\gammatrans}[1]{\stackrel{#1}{\hookrightarrow}}
\newcommand{\gammatransprime}[3]{{#1} \gammatrans{#2} {#3}}
\newcommand{\gammatransx}[1]{\stackrel{#1}{\hookrightarrow^*}}
%\newcommand{\trans}[3]{{#1} \stackrel{#2}{\longrightarrow} {#3}}
\newcommand{\run}[3]{{#1} \stackrel{#2}{\Longrightarrow} {#3}}
\newcommand{\sosrule}[2]{\frac{\begin{array}{c}#1\end{array}}{#2}}
\newcommand{\hoare}[3]{\{#1\}\ {\bf #2}\ \{#3\}}

\newcommand{\derivone}[3]{\mbox{\infer[#3]{#2}{#1}}}
\newcommand{\derivtwo}[4]{\mbox{\infer[#4]{#3}{{#1} & {#2}}}}
\newcommand{\derivdot}[4]{\mbox{\infer[#4]{#3}
{\begin{array}{l} {#1} \\ \vdots \\ {#2} \end{array}}}}

% comments, paragraphs
\newcommand{\comment}[1]{\vspace{\baselineskip}\noindent\fbox{
\parbox[b][\height][s]{120mm}{#1}}}
\newcommand{\mypar}[1]{\vspace*{\baselineskip}\noindent\emph{#1}}

% theories
\newcommand{\arith}{\langle \zed, +, \cdot, 0, 1 \rangle}
\newcommand{\hilbert}{\langle \zed, +, \cdot, 0, 1 \rangle^\exists}
\newcommand{\presbg}{\langle \zed, \geq, +, 0, 1 \rangle}
\newcommand{\divides}{\langle \zed, +, |, 0, 1 \rangle}
\newcommand{\lipshitz}{\langle \zed, \geq, +, |, 0, 1 \rangle^\exists}
\newcommand{\dioph}[1]{\mathfrak{D}(#1)}

% proofs
\newif\ifLongVersion\LongVersionfalse
%\newcommand{\proof}[1]{\noindent {\em Proof}: {#1}}
%\renewcommand{\qed}{$\Box$\vspace*{\baselineskip}}
%\newcommand{\qed}{$\Box$\vspace*{\baselineskip}}
\newcommand{\inproof}[1]{\ifLongVersion \proof{#1} \fi}
%\newcommand{\outproof}[2]{\ifLongVersion \else \vspace*{\baselineskip}\noindent Proof of {#1}: \proof{#2} \fi}


\documentclass[compress]{beamer}
%\documentclass{beamer}

\usepackage{beamerthemeJuanLesPins}
\usepackage{amsfonts,amsmath,amsthm,amssymb}
\usepackage{graphicx}
\usepackage{epsfig}
\usepackage{listings}
\usepackage[latin1]{inputenc}


\newcommand\xor{\oplus}
\newcommand{\emath}[1]{\ensuremath{#1}}
  \newcommand\tred[1]{\textcolor{red}{#1}}
  \newcommand\tblue[1]{\textcolor{blue}{#1}}
\newcommand\tgreen[1]{\textcolor{green}{#1}}
%\newcommand\tred[1]{\textcolor{red}{#1}}

\newcommand{\gspace}[0]{\hspace{2mm}}

\newcommand{\gnt}[1]{\langle \textit{#1} \rangle}
\newcommand{\gntsp}[1]{\gnt{#1} \gspace}
\newcommand{\gntm}[1]{$\gnt{#1}$}

\newcommand{\bbb}[1]{\mathbf{#1}}

\newcommand{\ga}[0]{::=}
\newcommand{\gam}[0]{$\ga$}

%\newcommand{\glit}[1]{\textrm{'}{\texttt{#1}}\textrm{'}}
%\newcommand{\glit}[1]{{\textbf{'}\textbf{#1}\textbf{'}}}
\newcommand{\glit}[1]{{\textbf{#1}}}
\newcommand{\glitsp}[1]{\glit{#1} \gspace}
\newcommand{\glitm}[1]{$\glit{#1}$}

\newcommand{\lparc}[0]{\textbf{'$\{$'}}
\newcommand{\rparc}[0]{\textbf{'$\}$'}}
\newcommand{\lparcsp}[0]{\lparc \gspace}
\newcommand{\rparcsp}[0]{\rparc \gspace}

\newcommand{\garrow}[0]{\textbf{'}\texttt{->}\textbf{'}}
\newcommand{\garrowsp}[0]{\garrow \gspace}

\newcommand{\gor}[0]{\\ & | &}
\newcommand{\gornb}[0]{\gspace | \gspace} % no break

\newcommand{\gsec}[1]{\noindent\textbf{#1}}

\newcommand{\spaceaptr}[0]{\hspace{1mm}}





%%%%%%%%%%%%%%%%%%%%%%  SSL RELATED STUFFS %%%%%%%%%%%%%%%%%%%%%%%%%%
%%%%%%%%%%%%%%%%%%%%%%%%%%%%%%%%%%%%%%%%%%%%%%%%%%%%%%%%%%%%%%%%%%%%%

\newcommand{\bi}[1]{{}^\omega\!{#1}^\omega}
\newcommand{\rbr}{{\bf ]\!]}}
\newcommand{\lbr}{{\bf [\![}}


\newcommand{\sem}[1]{\lbr #1 \rbr}

\newcommand{\arrow}[2]{\xrightarrow[{\scriptstyle #2}]{{\scriptstyle #1}}}

\newcommand{\defeq}[0]{:=}
\newcommand{\syntaxeq}[0]{::=}

\newcommand{\Dom}[1]{\mathcal{D}om(#1)}

\newcommand{\rrule}[2]{\lstinline!#1!  \to \lstinline!#2!}
\newcommand\Addr{\mathcal{A}ddr} %The set of heap addresses
\newcommand\LVar{\mathcal{L}Var} %The set of location variables
\newcommand\PVar{\mathcal{P}Var} %The set of pointer variables.

\newcommand{\Pp}[1]{PP(#1)} %% The set of the atomic propositions that appear inside a formula.
\newcommand{\Sp}[1]{SP(#1)} %% The set of spacial propositions that appear insde a formula.


%Size of a formula.
\newcommand{\sizef}[1]{|{#1}|}
%subformula
\newcommand{\subf}[0]{\triangleright}

\newcommand{\gton}[1]{>_{#1}}
\newcommand{\gtlvar}[0]{\gton{\LVar}}
\newcommand{\gtptrvar}[0]{\gton{\PVar}}

\newcommand{\Locvar}[1]{Locvar(#1)} %The set of location variables appearing
	%in a formula
\newcommand{\Ptrvar}[1]{Ptrvar(#1)} %same as abovem but for Ptr variables.

\newcommand{\NULL}[0]{\mbox{\texttt{NULL}}}

\newcommand{\Alloc}[1]{alloc(#1)}
\newcommand{\Emp}{\lstinline{Emp}}

\newcommand{\Unsep}[0]{\ast}
\newcommand{\Sep}[2]{#1 \Unsep #2}
\newcommand{\Sepf}[2]{ #1 \star #2}

\newcommand{\EqLoc}[2]{ (#1 = #2) }
\newcommand{\Andpure}[2]{ #1 \wedge #2}
\newcommand{\Pointsto}[2]{ #1 \mapsto #2}
\newcommand{\nil}[0]{\lstinline!nil!}
\newcommand{\Pointstonil}[1]{ #1 \mapsto \mbox{\nil}}
\newcommand{\Formula}[2]{ #1  \updownarrow #2}
\newcommand{\Exists}[2]{ \exists #1. #2}

\newcommand{\Substin}[3]{#1\lbrack #2 \gets #3 \rbrack}
\newcommand{\NForm}[1]{\mathcal{NF}(#1)}

\newcommand{\Ran}[1]{\mathcal{R}an(#1)}

\newcommand{\Vars}[1]{\mathcal{V}ars(#1)}
\newcommand{\FVars}[1]{\mathcal{F}Vars(#1)}


\newcommand{\Equa}[1]{\mathcal{E}q(#1)}
\newcommand{\Aff}[1]{\mathcal{A}ff{#1}}
\newcommand{\Nullptr}[1]{\mathcal{N}ullptr{#1}}


\newcommand{\Subst}[3]{#3\lbrack #1\leftarrow #2 \rbrack}
\newcommand{\fundec}[5]{\begin{array}{llll} {#1}:&{#2} &\mapsto & {#3} \\ &{#4} & \mapsto& #5  \end{array}}


% add/replaces A.P. #2 in formula #1.
\newcommand{\Addrep}[2]{#1 \lbrack #2 \rbrack}


%%%%%%%%%%% Compute new name for existential variables %%%%%%%%

\newcommand{\renfresh}[2]{Renfresh(#1,#2)}



%%%%%%%%%% Order Relation %%%%%%%%%%%%%%%%%%%%%%%%%%%
\newcommand{\LocOrderIneq}[0]{\sqsubset_{I}}
\newcommand{\LocOrderIeq}[0]{\sqsubseteq_{I}}
\newcommand{\PtrOrderIneq}[0]{\sqsubset_{S}}
\newcommand{\PtrOrderIeq}[0]{\sqsubseteq_{S}}
\newcommand{\ModOrderIeq}[0]{\sqsubseteq_{I,S}}
\newcommand{\ModOrderIneq}[0]{\sqsubset_{I,S}}


\newcommand{\true}{true}
\newcommand{\false}{false}
\newcommand{\dk}{dk}
%% #1 var name, #2 base, size, #4 offset
\newcommand{\ptrexp}[3]{\left ( \begin{array}{l} 
\ptrbase{#1}=#2 \\ 
\ptroffset{#1}=#3
\end{array} \right )}


\newcommand{\base}[2]{base_{#1}(#2)}
\newcommand{\samebase}[3]{Same\_base_{#1}({#2},{#3})}

\newcommand{\lbase}[1]{#1_{base}}
\newcommand{\lsize}[1]{#1_{size}}

\newcommand{\locvar}[3]{
\left ( \begin{array}{l}
\lbase{#1}=#2 \\
\lsize{#1}=#3
\end{array} \right )
}
%%%%%%%%%%%%% Hoar triples and sematic transformations %%%%%%%%%%%%%%%%%

%     instr  
% {P} -----> {Q}
%     effect
\newcommand{\transmodel}[4]{\lbrace #1 \rbrace \arrow{#2}{#3} \lbrace #4 \rbrace }

\newcommand{\Pre}[0]{\lbrace P \rbrace}
\newcommand{\Post}[0]{\lbrace Q \rbrace}

%%%%%%%%%%%% Footprint related stuff %%%%%%%%%%%%%%%%%%%%%%%%%%%%


%%%%%%%%%%%%%%%%%%%%%%%%%%%%%%%%%%%%%%%%%%%%%%%%%%%%%%%%%%%%%%



\newcommand{\ptrbase}[1]{{#1}_{base}}
\newcommand{\ptrsize}[1]{{#1}_{size}}
\newcommand{\ptroffset}[1]{{#1}_{off}}
\newcommand{\gmallocid}[0]{mid}


\newcommand{\interpretZ}[1]{\sem{#1}_{\mathbb{Z}}}
\newcommand{\interpretI}[1]{\sem{#1}_{I}}
\newcommand{\interpretP}[1]{\sem{#1}_{P}}
\newcommand{\interpretC}[1]{\sem{#1}_{C}}

\newcommand{\Eval}[1]{Eval(#1)}

%%%%%%%%%%%%%%%%%%% Validity interpretation compters 
\newcommand{\ival}[1]{{#1}_{\mbox{valid}}}

%%%%%%%%%%%%%%%%%%%%%%%%%%%%%%%%%%%%%%%%%%%%%%%%%%%%%%%%%%%%%%%%%%%%
% Inline text typology and other short notation definitions        %

\newcommand{\nth}[1]{$#1^{\mbox{th}}$}
\newcommand{\rd}[1]{$#1^{\mbox{rd}}$}


\newcommand{\syneq}[0]{::=} %Syntatic definition
\newcommand{\seq}[0]{\equiv} %Syntactic equality/equivalence

%%%%%%%%% Arithmetic operations
\newcommand{\intval}[2]{\left ( \begin{array}{l} #1 \\ #2 \end{array} \right )}

\newcommand{\valid}[1]{{#1}_{\mbox{\lstinline!valid!}}}
\newcommand{\intptr}[2]{\left ( \begin{array}{l} {#1} \\ {#2} 

\end{array} \right )}

\newcommand{\segfault}[0]{\mbox{\lstinline!segfault!}}

\newcommand{\andmath}[0]{\mbox{and}}
\newcommand{\congmod}[3]{#1 \equiv #2 \lbrack #3 \rbrack}
\newcommand{\notcongmod}[3]{#1 \not\equiv #2 \lbrack #3 \rbrack}
\newcommand{\isvalid}[2]{Valid_{#1}(#2)}
\newcommand{\DK}[2]{DK_{#1}(#2)}
%\newcommand{\interpa}[2]{Tr_{#1}(#2)}
\newcommand{\interpa}[2]{\sem{#2}_{#1}}

\newcommand{\interpacomp}[2]{Comp_{#1}(#2)}
\newcommand{\FRAMAC}{FRAMA-C}
\newcommand{\sizeof}[1]{\mbox{\lstinline!sizeof!}(#1)}

%% Counter associated to an interger variable.
\newcommand{\icnt}[1]{#1_{\mbox{cnt}}}

%% Access to memory @ pointer address P :
\newcommand{\accs}[1]{\mbox{\lstinline!access!}(#1)}



%%%%%%%%%%%%%%%%%% SSL RELATED STUFFS END BLOCK %%%%%%%%%%%%%%%%%%%%
%%%%%%%%%%%%%%%%%%%%%%%%%%%%%%%%%%%%%%%%%%%%%%%%%%%%%%%%%%%%%%%%%%%%  
  
  

%  \newcommand\bo[1]{\mathcal{#1}}
%  \newcommand\mA{\bo{A}}
%  \newcommand\terminemachin{+a.s. }
%  \newcommand\truc\terminemachin
%  \newtheorem{proposition}{Proposition}
%  \newtheorem{condition}{Condition}
%  \newtheorem{exemple}{Exemple}
%  \newtheorem{theoreme}{Theor�me}
%
%  \newtheorem{notation}{Notation}
%  \newcommand\notat[2]{\begin{condition}[#1] #2 \end{condition}}
%
%  \newcommand\cond[2]{\begin{condition}[#1] #2 \end{condition}}
%  \newtheorem{remarque}{Remarque}
  \usepackage{latexsym}
  \usepackage{graphicx}

%\input{../macros}

  \title{The Flatac Frama-c front-end}
 % \part{Main}


  \author{Radu Iosif, Florent Garnier}

\begin{document}
\frame{\titlepage 
}

\date{Febuary 2012}
%\part{Main}


\begin{frame}
\frametitle{Outline}
 \tableofcontents[part=1,pausesections]
\end{frame}



\section{Introduction }
\subsection{What is flatac}
\frame{
\frametitle{What is flatac ?}
\begin{itemize}
\item Part of a toolchain that aims at proving that C programs don't generate memory faults and don't violates assertions.
\item A front end that generates NTS based models of C programs.
\item Coded as a Frama-C plugin. 
\end{itemize}
}
\subsection{Tracking memory faults}
\frame{
\frametitle{Typical memory faults :}

\begin{itemize}
\item Memory access outside and allocated memory zone of the heap 
\item Access to an array outside of its bounds
\item Memory access using a non aligned address
\item Double free
\item Freeing an allocated segment using an pointer that does not points at the begining of the segment.
\item Memory leaks
\end{itemize}
}

\frame{
\frametitle{Two subkinds of properties :}
\begin{itemize}
\item Properties concerning the memory shape (Simple Separation Logic):
	\begin{itemize}
	\item Relation between pointer variables (Stack) and location variables (heaps).
	\item Memory allocation.
	\item Allocated Segment separation.

	\end{itemize}
\item Arithmetic properties :
	\begin{itemize}
	\item Memory segment access within its bounds. 
	\item Memory address alignment (Congruence).
	\end{itemize}
\end{itemize}
}

\frame{
\frametitle{Tracked property}
\tgreen{This front end aims at proving that C programs :}
\begin{itemize}
\item Have no execution run that lead to memory fault.
\item Have no exectution that violates some assertion expressed
using arithmetic constraints.
\end{itemize}
\tgreen{Flatac plugin: Front end of NTS error state reachability analysis }
\begin{itemize}
\item Extracts models of C Programs using Abstract Interpretation Techniques. 
\item Adds \tred{Numerical Transitions Systems} informations on the
model for \tred{a posteri Verification Phase}.  

\end{itemize}
}






\frame{
\frametitle{How to do that ?}
\begin{enumerate}
\item Extracting an extended cfg from Frama-c cfg $($Cil statements $\times$SSL memory abstractions$)^2$
\item Labelling the Ecfg transitions with Numerical Transition System expression --Guards, counter affectation and Function Calls.
\item If a SSL Abs value of a state is $\bot$, define this state as
an error state.
\item Export the labelled Ecfg into Nts Format.
\item Ask an analysis tool --Flata, Eldarica, to check whether 
some error state is reachable from the entry point (main function).

\end{enumerate}

}



\subsection{Flata-c a link in a toolchain}
\frame{
\frametitle{Flatac in the tool-chain:}
\begin{center}
\resizebox{9.5cm}{7cm}{\input{toolchain.pstex_t}}
\end{center}


}


\section{Abstract reprentation of the memory}
\subsection{The SSL logic}
\frame{
\frametitle{Abstract interpretation preliminary part}
 Simple Separation Logic formulae : Abstract domain.

$$
\begin{array}{lclr}
\phi & := & \Formula{\pi}{\sigma} \gspace | \gspace \Exists{l}{\phi} & \mbox{Formulae} \\
\pi & := & \Pointsto{x}{l} \gspace | \gspace \Pointstonil{x} \gspace | \gspace \EqLoc{l_1}{l_2} \gspace | \gspace \Andpure{\pi_1}{\pi_2} & {\mbox{Pure part}} \\
\sigma &:=& \Emp  \gspace | \gspace \Alloc{l} \gspace | \gspace \Sep{\sigma_1}{\sigma_2}\ \bot & \mbox{Spatial part}
\end{array}
$$

}



\frame{
\frametitle{Properties of SSL}

The problem that follows are decidable :
\begin{itemize}
\item Satisfiability (Valid configuration)
\item Entailment, Equivalence.
\item Memory leaks
\end{itemize}

Those problems are solved using rewriting techniques.
}

\frame{
\frametitle{Example of SSL formulae}
\begin{itemize}
\item $\Formula{ \Andpure{\Andpure{\Pointsto{x}{l_1}}{\Pointsto{y}{l}}}{\Pointstonil{z}}}{\Emp}$
\item $\Formula{\Andpure{\Andpure{\Pointsto{x}{l_1}}{\Pointsto{y}{l}}}{\Pointstonil{z}}}{\Alloc{l_1}}$
\item $\Formula{ \Andpure{\Andpure{\Pointsto{x}{l_1}}{\Pointsto{y}{l}}}{\Pointstonil{z}}}{\Sep{\Alloc{l_1}}{\Alloc{l}}}$
\item $\Formula{ \Andpure{\Andpure{\Pointsto{x}{l_1}}{\Pointsto{y}{l}}}{\Pointstonil{z}}}{\Sep{\Alloc{l_1}}{\Alloc{l}}}$
\item $\Formula{ \Andpure{x=y}{\Andpure{\Pointsto{x}{l_1}}{\Pointsto{y}{l}}}}{\Sep{\Alloc{l_1}}{\Alloc{l}}}$ (Unsat)
\item $\Formula{ \Andpure{x=y}{\Andpure{\Pointsto{x}{l_1}}{\Pointstonil{y}}}}{\Alloc{l_1}}$ (Unsat)
\item $\Formula{ true }{\Alloc{l}}$ (Leak)
\end{itemize}
}


\subsection{Memory model}
\frame{
\frametitle{Flata-c Memory model}
A memory model that associates counters to SSL
variables :

$$
\begin{array}{lcr}
\mbox{SSL Variable} & \mbox{NTS counter} & \\
x\in PVar & \mbox{\texttt{x\_offset}} & \mbox{offset} \\
l\in LVar & \mbox{\texttt{l\_size}} & \mbox{segment size}
\end{array}
$$
In order to :
\begin{itemize}
\item Associate to segment their size.
\item Associate to pointer their offset.
\item To express guards on memory access.
\end{itemize}
}

\frame{
\frametitle{Example}
$$
\begin{array}{lll}
\mbox{C "statement"} & \mbox{SSL formula}& \texttt{NTS transition} \\
\mbox{\texttt{int *x;}} &\Formula{\exists l \Pointsto{x}{l}}{\Emp}&\texttt{offset\_x'=0}\\
\mbox{\texttt{x=malloc(10);}} &\Formula{\exists l \Pointsto{x}{l}}{\Alloc{l}}&\texttt{size\_l=10}\\
\mbox{\texttt{x++;}} & \Formula{\exists l \Pointsto{x}{l}}{\Alloc{l}}& \texttt{offset\_x'+=sizeof(int)} \\
\mbox{\texttt{int y =*x;}} & \Formula{\exists l \Pointsto{x}{l}}{\Alloc{l}}& \texttt{offset\_x'} < \mbox{size\_l} \\
&\Formula{\exists . l \Pointsto{x}{l}}{\Alloc{l}}& \mbox{havoc(y)}
\end{array}
$$

}

\frame{
\frametitle{Example}
$$
\begin{array}{lll}
\mbox{C "statement"} & \mbox{SSL formula}& \texttt{NTS transition} \\
\mbox{\texttt{int *x;}} &\Formula{\exists l \Pointsto{x}{l}}{\Emp}&\texttt{offset\_x'=0}\\
\mbox{\texttt{x=malloc(10);}} &\Formula{\exists l \Pointsto{x}{l}}{\Alloc{l}}&\texttt{size\_l=10}\\
\mbox{\texttt{x++;}} & \Formula{\exists l \Pointsto{x}{l}}{\Alloc{l}}& \texttt{offset\_x'+=sizeof(int)} \\
\mbox{\texttt{int y =*x;}} & \Formula{\exists l \Pointsto{x}{l}}{\Alloc{l}}& \texttt{offset\_x'}<\mbox{size\_l} \\
&\Formula{\exists . l \Pointsto{x}{l}}{\Alloc{l}}& \mbox{havoc(y)}\\
\mbox{\texttt{x+=10;}} & \Formula{\exists l \Pointsto{x}{l}}{\Alloc{l}}& \texttt{offset\_x'+=10*sizeof(int)} \\
\end{array}
$$
}



\frame{
\frametitle{Example}
$$
\begin{array}{lll}
\mbox{C "statement"} & \mbox{SSL formula}& \texttt{NTS transition} \\
\mbox{\texttt{int *x;}} &\Formula{\exists l \Pointsto{x}{l}}{\Emp}&\texttt{offset\_x'=0}\\
\mbox{\texttt{x=malloc(10);}} &\Formula{\exists l \Pointsto{x}{l}}{\Alloc{l}}&\texttt{size\_l=10}\\
\mbox{\texttt{x++;}} & \Formula{\exists l \Pointsto{x}{l}}{\Alloc{l}}& \texttt{offset\_x'+=sizeof(int)} \\
\mbox{\texttt{int y =*x;}} & \Formula{\exists l \Pointsto{x}{l}}{\Alloc{l}}& \texttt{offset\_x'}<\mbox{size\_l} \\
&\Formula{\exists . l \Pointsto{x}{l}}{\Alloc{l}}& \mbox{havoc(y)}\\
\mbox{\texttt{x+=10;}} & \Formula{\exists l \Pointsto{x}{l}}{\Alloc{l}}& \texttt{offset\_x'+=10*sizeof(int)} \\
\mbox{\texttt{*x=42;}} & \bot & \tred{\texttt{offset\_x}\geq\mbox{size\_l}} \\
\end{array}
$$

\tred{Access to \texttt{*x} is out of bounds of allocated segment at \texttt{l}.}
}

\section{Model extraction}

\frame{
\frametitle{Cil representation of C programs}

Cil provides and AST and CFG info from C files.

\begin{block}{Most relevant information}
For each function of the AST :
\begin{itemize}
\item Expressions.
\item Locals and formal variables.
\item Statements of the Control flow graph.

\end{itemize}
\end{block}

}





\frame{
\frametitle{Control flow statemnt v.s. basic instructions}

\begin{block}{Control flow statement}
\begin{itemize}
\item \texttt{if(expr,blockif,blockelse)}
\item \texttt{switch(expr,case\_list)}
\item \texttt{while(expr)}
\end{itemize}
\end{block}

\begin{block}{Instruction statement}
\begin{itemize}
\item \texttt{lval=expr}
\item \texttt{lval=funcall(name,exp list)}
\end{itemize}
\end{block}
}

\frame{
\frametitle{Model extraction :}

Input : Cil AST and Control flow graph

Generated Model : Extended CFG,
  
$(S_i,S_f,S_{err},S,\to \in (S\times R \times S)) $

where :
\begin{itemize}
\item $S \in  ($Cil\_types.stmt$\times \ Abs)$,
\item $Abs =$ Set of SSL formula $\bigcup \bot$, 
\item $R$ is a set of possibly guarded NTS transitions. 
\end{itemize}
}


\frame{
\frametitle{Memory access rules}
Correct access:
{\tiny
$$
\begin{array}{|l|}
\hline
\transmodel{\phi}{\accs{P}}{
\begin{array}{l}
0 \leq \interpa{\phi}{x}+\sizeof{\tau}\leq \lsize{l} \\
 \wedge \congmod{(\interpa{\phi}{P}+\base{\phi}{P})
}{0}{\sizeof{\tau}}
\end{array}
}{\phi} \gspace \Alloc{l}\in \Sp{\phi} \gspace \mbox{where } \left \lbrace \begin{array}{l} P:\tau *, \\l\seq \base{\phi}{P}.\\ \end{array} \right.\\
\hline
\end{array}
$$
}

Access to an unallocated address :
{\tiny
$$
\begin{array}{|l|}
\hline
\transmodel{\exists l. \phi}{\accs{P}}{}{\bot} \gspace  \Alloc{l}\not \in \Sp{\phi}, l \in \FVars{\phi}, \mbox{ where } l \seq \base{\phi}{P} \\
\hline
\end{array}
$$
}
Access outside of an allocated zone, or with an unaligned address :
{\tiny
$$
\begin{array}{|l|}
\hline
\transmodel{\phi}{\accs{P}}{
\begin{array}{l}
\phantom{\vee}  \interpa{\phi}{P}+\sizeof{x} > \lsize{l} \\
\vee \notcongmod{\interpa{\phi}{P}}{0}{\sizeof{\tau}}
\end{array}
}{\bot} \gspace \Alloc{l}\in \Sp{\phi} \left \lbrace  \begin{array}{l} \mbox{where } P:\tau *,\\ l\seq\base{\phi}{P},\\ \end{array} \right . \\
\hline
\end{array}
$$
}


}

\frame{
\frametitle{Among other things}
\begin{itemize}
\item Validity of integer values : Initialization, difference
between two pointers, (Valid, Not Valid, Don't Know)
\item Transitions not generated when guards can be statically
proved false.
\end{itemize}

}


\frame{
\frametitle{Verification Phase : Reachability Analysis}

\begin{itemize}
\item Exporting the Ecfg Hierarchical Numerical Transition System.
\item Reachability analisys of the error states by FLATA and/or ELDARICA
\item If some error state is reachable : An alarm is raised.
\item If no error states is reachable : The program is free of the memory fault we consider. 
\end{itemize}
}

%\fram
%\frame{
%\frametitle{How to label terms with A.P. ?}
%\begin{center}
%\resizebox{10cm}{!}{\input{LabelLucas.pstex_t}}
%\end{center}
%}
%
%\frame{
%\frametitle{Precision of computational sets}
%\begin{center}
%\resizebox{8cm}{!}{\input{Precision.pstex_t}}
%\end{center}
%}
%

\end{document}

