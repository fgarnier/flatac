%This file enunciate the problematic of the biabduction in the realm of SSL logic
% and then decribes the rules and the algorithm that allows to solve the problem,
% i.e. computing specs and generating safety pre conditions and the associated
%post-conditions.

In this work, we intend to use SSL formulae to define specifications of functions,
in a way which is very close to the Hoar triples specifications.


\begin{definition}[SSL specification]
A SSL specification associated to a function \lstinline!foo!, is a tuple
of SSL formulae $\Pre, \Post$ such that if the memory model $<i,s,h>\models \Pre$
before a call to \lstinline!foo!, then $<i,s,h>\models \Post$ after \lstinline!foo!
returns.
It is denoted, using the usual Hoar triple fashion :

$$
\Pre \mbox{\lstinline!foo!} \Post.
$$
\end{definition}

TODO :
Expliquer et prouver que $\Post$ est une fonction de $\Pre$, exprimer
l'algo permettant de le calculer en utilisant les règles présentées
dans la section \ref{sec:memalloc}

Définir la relation d'ordre $\subf$, monter vérifier s'il existe unique
plus petit élement, de plus vérifier la correspondance avec les ordres
définis sur les modèles.

\begin{definition}[Safety specification]
The safety specification of a function \lstinline!foo! is the tuple
$\Pre \Post$, such that $\Pre$ is the smallest (quantifier free ?) SSL 
formula w.r.t. the subformula $\subf$ relation, such that $\Post\neq \top$. 
\end{definition}

\subsection{ Formulating the Bi-abduction problem}
The bi-abduction problem has been formulated in \cite{BiAbduction09}, and we reformulate this question in the realm of the SSL. We will see that the question is
easier two answer that in the aforementionned work, as the SSL logic is far
less expressive than the version of the separation logic that is considered
in their work. We don't express properties that deals, for instance, with list segments.

At some point in a program, the control flow reaches the function \lstinline!foo!,
and at this point the abstraction of the heap and the stack is a model
for the SSL formula $C$, i.e. in other words, $<i,s,h>\models C$.

Knowing the specifications of $\Pre \mbox{\lstinline!foo!} \Post$, is it possible
to decide wheter there exists two SSL formulae $A$ and $F$ such that 
$$\Sepf{A}{C} \vdash \Sepf{P}{F},$$
and compute them if they indeed exist.

Solving this question and providing an algorithm that computes $A$ and $F$ whenever
possible, would allow to answer the following questions :
\begin{itemize}
\item Is there a way to call the function \lstinline!foo! without generating a
memory fault ?
\item If yes, computing $A$ and $F$ would guarantee that the precondition $P$
is satisfied, and $F$ represent an invarient. 
\end{itemize} 

In a few words, computing $A$ consists in adding to $C$ what is missing to 
get $P$, and $F$ is an abstraction of the heap that won't be affected the call to \lstinline!foo!, which is in fact an invarient usually called the "frame".
However proving that the equation is unsatisfiable is equivalent to prove
that the program is not safe.
Here, we are looking forward computing the smallest $A$ that satifies the bi-abduction problem.

\textbf{Algorithme candidat :}
The figure below brings a visual clue about the meaning of the biabduction problem :

\begin{figure}[!htbf]
\label{fig:biabdprinciple}
\begin{center}
\resizebox{10cm}{!}{\input{./biadcandidate.pstex_t}}
\end{center}
\caption{Bi-abduction at a glance}
\end{figure}

If one succeed to find some SSL formula $A$ in a separated way to the current
context $C$, so that the specification $P$ is a subformula of $\Sepf{A}{C}$ then,
the function \lstinline!foo! can be called if one guarantee $A$, as  in this case the precondition of \lstinline!foo! is satisfied by a subformula.
The $F$ formula, i.e. the frame, is an invarient part of the heap, and will taken
under consideration to apply the frame rule to compute the heap after the call
to \lstinline!foo! completes.

\subsection{Solving the bi-abduction problem}

First ensure that both $C$ and $P$ are un normal form, if not assign to $C$ and
$P$ their respective normal form.
The basic idea consists in reducing the term/proposition $C \vdash P$ using the
rewrite/inference rules presented in section \ref{sec:entailement}.
 
\begin{figure}[!htbf]
\label{fig:biabdprinciple}
\begin{center}
\resizebox{10cm}{!}{\input{./biadcandidatealgo.pstex_t}}
\end{center}
\caption{Computing Frame and Antiframe}
\end{figure}


All litteral that are shared by $C$ and $P$ get eliminated. 
The reminder in the left formula is the frame and what appears in the right 
hand side is the Frame, that's to say what one need to guarantee to ensure that 
\lstinline!foo! will be called
without generating a memory fault.  

\begin{lemma}[(TODO) It terminates, sound and complete, bonnes propriétés et consort.]
\end{lemma}
\begin{proof}
\end{proof}

\begin{lemma}[(TODO) Check that]
$A$ is the smallest antiframe and $F$ is the largest invarient.
\end{lemma} 

\begin{theorem}[Bi-abductability]
\label{th:biabductable}
It is possible to call a function of spec $P$ in a context $C$, if and only if
the term $C\vdash P \leadsto F \vdash A $ such that $\Sepf{F}{A}$ is satisfiable.
\end{theorem}


Thanks to the theorem \ref{th:biabductable}, we define a sound and complete decision procedure that allows to solve the bi-abduction problem, that's to say :
\begin{itemize}
\item compute $A$ and $F$ is and only if they exist,
\item answer "UNSAT" when there is no such $A$ and $F$. 
\end{itemize} 



\textbf{Biabduction Algorithm :}
\begin{itemize}
\item Input : $(P,C)$ : Tuple of SSL Formulae
\item Output : $(A,F) \cup \lbrace UNSAT\rbrace$ : Tuple of SSL formulae or UNSAT.
\item Algorithm :
\begin{itemize}

\item Reduce $C\vdash P \leadsto F\vdash A$
\item Compute $J:=\NForm{\Sepf{A}{F}}$
\item If  there is no $(x,l)$ s.t. $\Sepf{\Pointsto{x}{l} \wedge \Pointstonil{x}}{\Alloc{l}}\not\in J$
and there is not $l$ such that $\Sep{\Alloc{l}}{\Alloc{l}}\in \Sp{J}$
\item Then return $(A,F)$
\item Else return $UNSAT$.
\end{itemize}
\end{itemize}

We shall prove the theorem~\ref{th:biabductable} before going further. 

\begin{proof}
Let us start to prove the implication. Consider that the current context is
abstracted by $C$ and that it is possible to safely call a function \lstinline!foo!
which pre condition is $P$. 


\end{proof}

\section{Computing specs using bi-abduction}
